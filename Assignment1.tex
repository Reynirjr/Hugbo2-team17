\documentclass{article}
\usepackage{booktabs}
\usepackage{longtable}
\usepackage{array}
\usepackage{geometry}
\geometry{a4paper, margin=1in}

\title{Assignment 1 - Project Plan and Requirements}
\author{Team [Number]}
\date{\today}

\begin{document}

\maketitle

\section{Project Vision}

% Provide your project vision document, data sheet, or press release here.
% If revising from HBV501G, adjust accordingly.

\subsection{1.5 Title Here} Skip subsections if using Data Sheet or Press Release. Note whether you are doing a new vision document, or revising a previous one.

\subsection{1.x Additional Section Title}

\subsection{2.x Additional Section Title (except 2.1)}

\section{Product Backlog}

Below is the prioritized backlog with user stories:
Note what kind of priority system is used. In order? grouped? Is a lower number higher priority? (Recommended to group and use lower number = higher priority; E.g. P1 is higher priority than P2).
\begin{itemize}
    \item \textbf{User Story 1 [US1; P1]}: Information about user story here
    \item \textbf{User Story 2 [US2; P2]}: Information about user story here
    % Add more user stories as needed
\end{itemize}

\section{User Story Estimates}

Below is the estimation table for user stories using the PERT formula:
Note down what size you're using for estimates. Is it person days, or work hours?
\begin{longtable}{|c|c|c|c|c|}
    \hline
    \textbf{User Story} & \textbf{Best Case} & \textbf{Most Likely} & \textbf{Worst Case} & \textbf{PERT Estimate} \\
    \hline
    US1 & X & Y & Z & $\frac{X + 4Y + Z}{6}$ \\
    \hline
    US2 & A & B & C & $\frac{A + 4B + C}{6}$ \\
    \hline
    % Add more user stories as needed
\end{longtable}

\section{Project Schedule}

Below is the schedule for the 10-week project timeline (starting the week after assignment 1 turn-in):
Note that there are 4 sprints, one for each assignment. Sprint 1 is two weeks, Sprint 2 and Sprint 3 are three weeks, and the final sprint is two weeks. Decide who is going to be the P.O. for each of the sprints.
\textbf{This is a template for a schedule, adjust as needed.}

\begin{longtable}{|c|c|c|c|c|c|}
    \hline
    \textbf{Week} & \textbf{User Stories} & \textbf{Expected Hours} & \textbf{P.O. (Initials)} & \textbf{Sprint} & \textbf{Consultation}\\
    \hline
    1 & None & XX & AB & 1 & \textbf{A1 Presentation}\\
    \hline
    2 & US1, Android skeleton & XX & AB & 1 & Model Drafts\\
    \hline
    3 & US2, US3 & XX & CD & 2 & \textbf{A2 Presentation}\\
    \hline
    4 & US4, US5, US6 & XX & CD & 2 & Dev support\\
    \hline
    5 & US7, US8 & XX & CD & 2 & Dev support\\
    \hline
    6 & US9, US10 & XX & EF & 3 & \textbf{A3 Presentation}\\
    \hline
    % Continue until week 10
\end{longtable}

\end{document}
